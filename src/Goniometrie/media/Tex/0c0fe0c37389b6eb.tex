\documentclass[preview]{standalone}

\usepackage[english]{babel}
\usepackage{amsmath}
\usepackage{amssymb}

\begin{document}

\begin{center}
These same principles can also be used to derive the values for the other half of the unit circle. But then you would subtract $\pi$ in the third quadrant and $1 \frac{1}{2} \pi$ in the last quadrant.
\end{center}

\end{document}
