\documentclass[preview]{standalone}

\usepackage[english]{babel}
\usepackage{amsmath}
\usepackage{amssymb}

\begin{document}

\begin{center}
Welke waarde bij welke hoek hoort kan je zelf achter halen door te kijken naar het begin- en eindpunt en dan te kijken naar de voortgang. Bij een grotere hoek gaan de $x$ en $y$ waardes meer lijken op de eindwaardes en minder op de startwaardes.
\end{center}

\end{document}
