\documentclass[preview]{standalone}

\usepackage[english]{babel}
\usepackage{amsmath}
\usepackage{amssymb}

\begin{document}

\begin{center}
Looking at the angle $\theta = \frac{2}{3} \pi$, we can use the same method. First subtract the starting angle $\frac{1}{2} \pi$ so that $\theta = \frac{1}{6} \pi$. This means that the least amount of progress is made to go from (0,1) to (-1, 0).
\end{center}

\end{document}
